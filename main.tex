\documentclass[11pt]{beamer}

\usepackage[spanish]{babel}

\usetheme[
  progressbar=frametitle,
]{metropolis}

\usepackage{tikz}
\usetikzlibrary{positioning}
\usetikzlibrary{decorations}
\usetikzlibrary{patterns}

\usepackage{pgfplots}
\pgfplotsset{compat=1.18}

\usepackage{multicol}

\title{El polinomio de Taylor}
\subtitle{Una introducción a la aproximación de funciones}
\author{Francisco González Rubio \and Pedro Pasalodos Guiral \and Mario Vago Marzal}
\date{Curso 2023--2024}
\institute{Universitat de València}

\begin{document}
  {
    % A workaround to hide an overfull \vbox warning produced by
    % the title frame of the Metropolis theme.
    \vfuzz=16pt
    \maketitle
  }

  \begin{frame}{Índice}
    \setbeamertemplate{section in toc}[sections numbered]
    \tableofcontents
  \end{frame}

  \section{El problema del péndulo}

  \section{El polinomio de Taylor}
    \begin{frame}{Polinomio de Taylor}

    \begin{block}{Problema de aproximación}
    
    Sea $f (x)$ una función $n$ veces derivable en un intervalo
    $[a, b]$. Queremos calcular $P_n(x) \in \Pi_n$, un polinomio
    que verifique
     
    $$P_n^{(k)}(x_0) = f^{(k)}(x_0), \quad k = 0, 1, \dots , n$$
    donde $x_0$ es un punto fijado del intervalo $]a, b[$.
    \end{block}
    
    \begin{block}{Teorema}
      El problema de aproximación tiene solución y es única.
    \end{block}
    
\end{frame}
    \begin{frame}{Forma natural del polinomio de Taylor}
  Si tomamos la base natural de $\Pi_n$, obtenemos la forma natural del
  polinomio de Taylor:
    
  \begin{center}
    \begin{tikzpicture}
      \node[rectangle, fill=TolLightBlue!20!white] {
        $\displaystyle P_n(x) = 
        \sum_{i=0}^n \frac{f^{(i)}(x_0)}{n!} (x-x_0)^i$
      };
      \end{tikzpicture}
  \end{center}

  \begin{multicols}{2}
    Tomando $f(x) = \sen x$ y $x_0 = 0$, tenemos que
    \begin{itemize}
      \item $P_1(x) = x$,
      \item $P_3(x) = x-\frac{x^3}{6}$,
      \item y $P_5(x) = x - \frac{x^3}{6} + \frac{x^5}{120}$.
    \end{itemize}

    \columnbreak

    \begin{center}
      \begin{tikzpicture}[scale=0.6, transform shape]
        \begin{axis}[
          mlineplot,
          no markers,
          axis lines=center,
          legend pos=south east,
          trig format=rad,
          thick,
        ]
          \addplot+[
            domain=-pi:pi,
            samples=100,
          ] {sin(x)};
            \addlegendentry{$\sen(x)$}
  
          \addplot+[
              domain=-pi:pi,
              samples=100,
            ] {x};
              \addlegendentry{$P_1(x)$}
      
          \addplot+[
            domain=-pi:pi,
            samples=100,
          ] {x - x^3/6};
            \addlegendentry{$P_3(x)$}
      
          \addplot+[
            domain=-pi:pi,
            samples=100,
          ] {x - x^3/6 + x^5/120};
            \addlegendentry{$P_5(x)$}
        \end{axis}
      \end{tikzpicture}
    \end{center}
  \end{multicols}
\end{frame}

    
  \section{El error de aproximación}
     \begin{frame}{Forma del error de Lagrange}
  \begin{center}
    \begin{tikzpicture}[scale=0.65, transform shape]
      \begin{axis}[
        mlineplot,
        no markers,
        axis lines=center,
        legend pos=south east,
        trig format=rad,
        thick,
      ]
        \addplot+[
          domain=-pi:pi,
          samples=100,
        ] {sin(x)};
          \addlegendentry{$\sen(x)$}
    
        \addplot+[
          domain=-2:2,
        ] {x};
          \addlegendentry{$x$}
        
        \def\errorx{pi/2}
        \node (val_real) at 
          (axis cs: \errorx, {sin(\errorx)}) {};
        \node (val_aprox) at 
          (axis cs: \errorx, \errorx) {};
        \draw[
          dashed,
          thin,
          draw=TolDarkRed,
          ->,
        ] (val_aprox.center) -- (val_real.center)
          node[midway, right] 
          {$\varepsilon_{\frac{\pi}{2}}$};
      \end{axis}
    \end{tikzpicture}
  \end{center}
  
  Si $f$ es $n + 1$ veces derivable, existe $\xi_x \in (x_0, x)$ tal que
  \begin{center}
    \begin{tikzpicture}
      \node[rectangle, fill=TolLightRed!20!white] {
        $\displaystyle f(x) - P_n (x) = \frac{f^{(n+1)} (\xi_x)}{(n + 1)!} (x - x_0)^{n+1}$
      };
    \end{tikzpicture}
  \end{center}
\end{frame}


  \section{Caracterización de los polinomios de Taylor}

  \section*{Conclusiones}

\end{document}

\documentclass{beamer}

\usepackage[spanish]{babel}

\usetheme[
  progressbar=frametitle,
  block=fill,
]{metropolis}
\usefonttheme[onlymath]{serif}

\usepackage{tikz}
\usetikzlibrary{positioning}
\usetikzlibrary{decorations}
\usetikzlibrary{patterns}

\usepackage{pgfplots}
\pgfplotsset{compat=1.18}

\usepackage{multicol}

\title{El polinomio de Taylor}
\subtitle{Una introducción a la aproximación de funciones}
\author{Francisco González Rubio \and Pedro Pasalodos Guiral \and Mario Vago Marzal}
\date{Curso 2023--2024}
\institute{Universitat de València}

\begin{document}
  {
    % A workaround to hide an overfull \vbox warning produced by
    % the title frame of the Metropolis theme.
    \vfuzz=16pt
    \maketitle
  }

  \begin{frame}{Índice}
    \setbeamertemplate{section in toc}[sections numbered]
    \tableofcontents
  \end{frame}

  \section{El problema del péndulo}
    \begin{frame}{El problema del péndulo}
  \begin{center}
    \begin{tikzpicture}[
      scale=1.4,
      transform shape,
      every node/.style={font=\tiny},
    ]
      \begin{scope}[yshift=5.25cm]
        \draw[
          pattern=north east lines,
          pattern color=TolDarkBlue,
        ] (-2.5,-0.25) rectangle (2.5,0.25);
      \end{scope}
  
      \begin{scope}[yshift=5cm]
        \draw[dashed, ultra thin] 
          (0,-2.15) -- (2,-2.15);
  
        \draw[dotted, thick, draw=TolLightGreen] 
          (-120:2.15) arc (-120:-60:2.15);
  
        \draw (0,0) -- (-60:2)
          node[midway, right] {$R$};
        \draw[fill=TolLightGreen] 
          (-60:2.15) circle (0.15);
  
        \draw[dashed, thin] (0,0) -- (-90:2.15);
  
        \draw[->, draw=TolLightBrown] 
          (-90:0.5) arc (-90:-60:0.5)
          node[midway, below, ] {$\theta$};
  
        \begin{scope}[xshift=0.25cm]
          \draw[|-|]
            (-60:2.15) -- (-60:2.15 |- -2.15,-2.15)
            node[midway,right,inner sep=0pt,xshift=0.1cm]
              {$h = R(1 - \cos \theta)$};
        \end{scope}
      \end{scope}
    \end{tikzpicture}
  \end{center}

  \vfill

  \begin{center}
    \begin{tikzpicture}[
      scale=0.9,
      transform shape,
      every node/.style={
        font=\small,
      },
    ]
      \begin{scope}[
        every node/.style={
          minimum height=1.1cm,
          rectangle,
        },
      ]
        \node[fill=TolLightRed!20!white]
          (exacta) at (0,0) {
          $h = R(1 - \cos \theta)$
        };

        \node[fill=TolLightGreen!20!white] 
          (aprox) [right=4cm of exacta] {
          $h \approx R \dfrac{\theta^2}{2}$
        };
      \end{scope}

      \draw[decorate, decoration={snake, amplitude=0.4mm}, <->] (exacta) -- (aprox)
        node[midway, below, yshift=-0.3cm] {
          $\cos \theta \approx 1 - \dfrac{\theta^2}{2}$
        };
    \end{tikzpicture}
  \end{center}
\end{frame}

    \begin{frame}{La aproximación del péndulo}
  \begin{center}
    \begin{tikzpicture}[scale=0.55, transform shape]
      \node (coseno) {
        \begin{tikzpicture}
          \begin{axis}[
            mlineplot,
            no markers,
            axis lines=center,
            trig format=rad,
            thick,
          ]
            \addplot+[
              domain=-pi:pi,
              samples=100,
            ] {cos(x)};
              \addlegendentry{$\cos \theta$}
          \end{axis}
        \end{tikzpicture}
      };
      
      \node (taylor) [right=4cm of coseno] {
        \begin{tikzpicture}
          \begin{axis}[
            mlineplot,
            no markers,
            axis lines=center,
            trig format=rad,
            thick,
          ]
            \addplot+[
              domain=-pi:pi,
              samples=100,
              dashed,
              thin,
            ] {cos(x)};
              \addlegendentry{$\cos \theta$}
      
            \addplot+[
              domain=-2:2,
            ] {1 - x^2/2};
              \addlegendentry{$1 - \frac{\theta^2}{2}$}
          \end{axis}
        \end{tikzpicture}
      };
  
      \draw[
        decorate,
        decoration={snake, amplitude=0.4mm}, 
        <->,
      ]
        (coseno.east) -- (taylor.west)
        node [midway, below, yshift=-0.25cm] 
        {$\cos \theta \approx 1 - \dfrac{\theta^2}{2}$};
    \end{tikzpicture}
  \end{center}

    \[
      \cos \theta \approx
      1 - \frac{\theta^2}{2} + \frac{\theta^4}{4!} - \frac{\theta^6}{6!} + \cdots
    \]
\end{frame}


  \section{El polinomio de Taylor}
    \begin{frame}{Polinomio de Taylor}
  \begin{alertblock}{Problema de aproximación de Taylor}
    Dada una función $f: [a, b] \to \mathbb{R}$ que es derivable $n$ veces en
    $x_0 \in ]a, b[$, se trata de encontrar un polinomio $P_n(x)$ de grado
    menor o igual que $n$ tal que
    \[
      P_n^{(k)}(x_0) = f^{(k)}(x_0), \quad k = 0, 1, \dots , n.
    \]
  \end{alertblock}
  
  \begin{exampleblock}{Solución del problema de aproximación de Taylor}
    El problema de aproximación de Taylor tiene solución y es única.
  \end{exampleblock}
\end{frame}

    \begin{frame}{Forma natural}
  \begin{block}{Forma natural}

    El polinomio de Taylor centrado en $x_0$ de grado $n$ se pude 
    escribir de la manera siguiente:
      
    \begin{center}
      \begin{tikzpicture}
        \node[rectangle, fill=TolLightBlue!20!white] {
          $\displaystyle P_n(x) = \sum_{i=0}^n \frac{f^{(i)}(x_0)}{n!} (x-x_0)^i$
          };
        \end{tikzpicture}
    \end{center}
  \end{block}

  \begin{multicols}{2}
    \begin{tikzpicture}[scale=0.6, transform shape]
  \begin{axis}[
    mlineplot,
    no markers,
    axis lines=center,
    legend pos=south east,
    trig format=rad,
    ]
    \addplot+[
      domain=-pi:pi,
      samples=100,
    ] {sin(x)};
      \addlegendentry{$\sen(x)$}

    \addplot+[
      domain=-pi:pi,
      samples=100,
    ] {x - x*x*x/6};
      \addlegendentry{$P_3(x)$}

    \addplot+[
        domain=-pi:pi,
        samples=100,
      ] {x - x*x*x/6 + x*x*x*x*x/120};
        \addlegendentry{$P_6(x)$}
     
  \end{axis}
\end{tikzpicture}
  \columnbreak
  \begin{itemize}
    \item $P_1(x) = x$
    \item $P_3(x) = x-\frac{x^3}{6}$
    \item $P_5(x) = x - \frac{x^3}{6} + \frac{x^5}{120}$
  \end{itemize}
\end{multicols}
        

\end{frame}
    
  \section{El error de aproximación}
    \begin{frame}{Error de Taylor}

    \begin{multicols}{2}
    
        \begin{figure}
            \centering
            \begin{tikzpicture}
                \begin{axis}[x=2.5cm, y=2.5cm,
                axis lines=middle,
                axis x line shift=0,
                xlabel=\(x\),ylabel=\(y\),
                x label style={anchor=north},
                y label style={anchor=east},
                xmin=-0.25,xmax=1.75,ymin=-0.25,ymax=1.75,
                xtick={1.25},
                xticklabels={$x_0$},
                ytick={-1},
                clip=false,smooth,
                samples = 100]
        
                    \addplot[beamerorange, thick, domain = -0.13:1.5]{sin(\x r)};
                    \draw[dashed, beamergreen] (-0.13,-0.13) -- (1.5,1.5);
                    \path (1.25,0.94898461) edge node[right] {$\varepsilon$} (1.25,1.25);
                    
                \end{axis}
            \end{tikzpicture}
        \end{figure}
        

  
    %\begin{block}{Teorema del Error}
  
    Sea $f(x)$ una función $n + 1$ veces derivable en $]a, b[$
    y $x_0 \in]a, b[$. Si $P_n(x)$ es el polinomio de
    Taylor de $f (x)$ en el punto $x_0$, entonces para cada $x \in]a, b[$
    existe $\xi_x \in]a, b[$ tal que   
  
    \begin{equation*}
        f(x) - P_n (x) = \frac{f^{(n+1)} (\xi_x)}{(n + 1)!} (x - x_0)^{n+1}
        \label{taylor}
    \end{equation*}
  
    %\end{block}
  
\end{multicols}
  
\end{frame}

  \section{Caracterización de los polinomios de Taylor}
    \begin{frame}{Primera caracterización}
    %\begin{proposition}
    Si $f(x)$ es $n+1$ veces derivable en un entorno de $x_0$,
    el polinomio de Taylor de $f(x)$ en $x_0$ es el único
    polinomio $P_n(x)$ de grau $n$ que cumple
    \\
    $\lim_{x \rightarrow x_0} \frac{f(x)-P_n(x)}{(x-x_0)^n}=0$.
    %\end{proposition}
\end{frame}

\begin{frame}{Ejemplo}
Vamos a ver que $P_(x)= \Sigma^{n}_{k=0} \frac{x^k}{k!}$ es el polinomio
de Taylor de $f(x)=e^x$ amb $x_0=0$.

$$\lim_{x \rightarrow 0} \frac{e^x-P_n(x)}{x^n}\underset{L'Hôpital}{=}
\lim_{x \rightarrow 0} \frac{e^x-P_{n-1}(x)}{nx^{n-1}}\underset{L'Hôpital}{=} \dots \underset{L'Hôpital}{=}$$
\\
$$\lim_{x \rightarrow 0} \frac{e^x-1}{n!}=0$$ 
\\

Por la caracterización anterior ya lo tenemos

    
\end{frame}

\begin{frame}{Segunda caracterización}
    Si $f(x)$ es $n+1$ veces derivable alrededor del punto $x_0$, el polinomio de Taylor de grado $n$ a $x_0$ es el único polinomio $P_n(x)$ que verifica una desigualdad del tipo
    \\
    $$|f(x)-P_n(x)| \leq M|x-x_0|^{n+1}$$
    com $M > 0$, en un entorno del punto $x_0$.
\end{frame}

\begin{frame}{Ejemplo}

    Vamos a ver que $P_n(x)= 1 - x^2 + x^4 - x^6 \dots + (-1)^nx^{2n}$ es polinomio de Taylor de
    $f(x)= \frac{1}{1+x^2}$ en $x_0=0$.

    $$|\frac{1}{1+x^2}-(1 - x^2 + x^4 - x^6 \dots + (-1)^nx^{2n})|
    =$$$$|\frac{1-1-x^2+x^2+x^4-x^4 \dots (-1)^{n+1}x^{2n+2}}{1+x^2}|=|\frac{(-1)^{n+1}x^{2n+2}}{1+x^2}|$$
    Si $|x|< \frac{1}{2}$:
    $$|\frac{x^{2n+2}}{1+x^2}|\leq |\frac{x^{n+1}}{1+x^2}| \leq 2|x^{n+1}| $$

    Esta desigualdad nos garantiza que $P_n(x)$
    es una caracterización
\end{frame}

    \begin{frame}{La función exponencial}
  Es habitual presentar $e$ como la serie
  \[
    e = \sum_{n=0}^\infty \frac{1}{n!}.
  \]

  Se debe a que el polinomio de Taylor de la exponencial es
  \[
    P_n(x) = \sum_{k=0}^n \frac{x^k}{k!}.
  \]

  Usando la caracterización anterior y la regla de L'Hôpital comprobamos
  que
  \[
    \lim_{x \to 0} \frac{e^x - P_n(x)}{x^n}
    = \frac{e^x - \sum_{k = 0}^{n - 1} \frac{x^k}{k!}}{nx^{n - 1}}
    = \ldots = 1.
  \]
\end{frame}

    \begin{frame}{Segunda caracterización}
  Si $f(x)$ es $n+1$ veces derivable alrededor del punto $x_0$, el polinomio de Taylor de grado $n$ a $x_0$ es el único polinomio $P_n(x)$ que verifica una desigualdad del tipo
  \\
  $$|f(x)-P_n(x)| \leq M|x-x_0|^{n+1}$$
  com $M > 0$, en un entorno del punto $x_0$.
\end{frame}

    \begin{frame}{Funciones generatrices}
  Las funciones generatrices son funciones muy útiles usadas en
  combinatoria, probabilidad y otras áreas.

  Una de tantas funciones generatrices útiles es
  \[
    f(x) = 1 - x^2 + x^4 - x^6 + \cdots,
  \]
  que resulta ser el polinomio de Taylor en $0$ (y esto es muy útil) de
  \[
    g(x) = \frac{1}{1 + x^2}.
  \]

  Usando la segunda caracterización, si $\lvert x \rvert < \frac{1}{2}$ y
  $M = 2$,
  \[
    \lvert f(x) - g(x) \rvert 
    \leq \left\lvert \frac{x^{n + 1}}{1 + x^2} \right\rvert
    \leq 2 \lvert x \rvert^{n + 1}.
  \]
\end{frame}


  \section*{Conclusiones}

\end{document}

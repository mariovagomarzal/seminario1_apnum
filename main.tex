\documentclass[11pt]{beamer}

\usepackage[spanish]{babel}
\usepackage{multicol}

\usetheme[
  progressbar=frametitle,
]{metropolis}

\usepackage{tikz}
\usetikzlibrary{positioning}
\usetikzlibrary{decorations}
\usetikzlibrary{patterns}

\usepackage{pgfplots}
\pgfplotsset{compat=1.18}


\title{El polinomio de Taylor}
\subtitle{Una introducción a la aproximación de funciones}
\author{Francisco González Rubio \and Pedro Pasalodos Guiral \and Mario Vago Marzal}
\date{Curso 2023--2024}
\institute{Universitat de València}

\begin{document}
  {
    % A workaround to hide an overfull \vbox warning produced by
    % the title frame of the Metropolis theme.
    \vfuzz=16pt
    \maketitle
  }

  \begin{frame}{Índice}
    \setbeamertemplate{section in toc}[sections numbered]
    \tableofcontents
  \end{frame}

  \section{El problema del péndulo}

  \section{El polinomio de Taylor}
  \begin{frame}{Polinomio de Taylor}
  \begin{alertblock}{Problema de aproximación de Taylor}
    Dada una función $f: [a, b] \to \mathbb{R}$ que es derivable $n$ veces en
    $x_0 \in ]a, b[$, se trata de encontrar un polinomio $P_n(x)$ de grado
    menor o igual que $n$ tal que
    \[
      P_n^{(k)}(x_0) = f^{(k)}(x_0), \quad k = 0, 1, \dots , n.
    \]
  \end{alertblock}
  
  \begin{exampleblock}{Solución del problema de aproximación de Taylor}
    El problema de aproximación de Taylor tiene solución y es única.
  \end{exampleblock}
\end{frame}

  \begin{frame}{Forma natural}
  \begin{block}{Forma natural}

    El polinomio de Taylor centrado en $x_0$ de grado $n$ se pude 
    escribir de la manera siguiente:
      
    \begin{center}
      \begin{tikzpicture}
        \node[rectangle, fill=TolLightBlue!20!white] {
          $\displaystyle P_n(x) = \sum_{i=0}^n \frac{f^{(i)}(x_0)}{n!} (x-x_0)^i$
          };
        \end{tikzpicture}
    \end{center}
  \end{block}

  \begin{multicols}{2}
    \begin{tikzpicture}[scale=0.6, transform shape]
  \begin{axis}[
    mlineplot,
    no markers,
    axis lines=center,
    legend pos=south east,
    trig format=rad,
    ]
    \addplot+[
      domain=-pi:pi,
      samples=100,
    ] {sin(x)};
      \addlegendentry{$\sen(x)$}

    \addplot+[
      domain=-pi:pi,
      samples=100,
    ] {x - x*x*x/6};
      \addlegendentry{$P_3(x)$}

    \addplot+[
        domain=-pi:pi,
        samples=100,
      ] {x - x*x*x/6 + x*x*x*x*x/120};
        \addlegendentry{$P_6(x)$}
     
  \end{axis}
\end{tikzpicture}
  \columnbreak
  \begin{itemize}
    \item $P_1(x) = x$
    \item $P_3(x) = x-\frac{x^3}{6}$
    \item $P_5(x) = x - \frac{x^3}{6} + \frac{x^5}{120}$
  \end{itemize}
\end{multicols}
        

\end{frame}
  \section{El error de aproximación}
    \begin{frame}{Error de Taylor}

    \begin{multicols}{2}
    
        \begin{figure}
            \centering
            \begin{tikzpicture}
                \begin{axis}[x=2.5cm, y=2.5cm,
                axis lines=middle,
                axis x line shift=0,
                xlabel=\(x\),ylabel=\(y\),
                x label style={anchor=north},
                y label style={anchor=east},
                xmin=-0.25,xmax=1.75,ymin=-0.25,ymax=1.75,
                xtick={1.25},
                xticklabels={$x_0$},
                ytick={-1},
                clip=false,smooth,
                samples = 100]
        
                    \addplot[beamerorange, thick, domain = -0.13:1.5]{sin(\x r)};
                    \draw[dashed, beamergreen] (-0.13,-0.13) -- (1.5,1.5);
                    \path (1.25,0.94898461) edge node[right] {$\varepsilon$} (1.25,1.25);
                    
                \end{axis}
            \end{tikzpicture}
        \end{figure}
        

  
    %\begin{block}{Teorema del Error}
  
    Sea $f(x)$ una función $n + 1$ veces derivable en $]a, b[$
    y $x_0 \in]a, b[$. Si $P_n(x)$ es el polinomio de
    Taylor de $f (x)$ en el punto $x_0$, entonces para cada $x \in]a, b[$
    existe $\xi_x \in]a, b[$ tal que   
  
    \begin{equation*}
        f(x) - P_n (x) = \frac{f^{(n+1)} (\xi_x)}{(n + 1)!} (x - x_0)^{n+1}
        \label{taylor}
    \end{equation*}
  
    %\end{block}
  
\end{multicols}
  
\end{frame}

  \section{Caracterización de los polinomios de Taylor}

  \section*{Conclusiones}

\end{document}
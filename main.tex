\documentclass[11pt]{beamer}

\usepackage[spanish]{babel}

\usetheme[
  progressbar=frametitle,
]{metropolis}

\usepackage{tikz}
\usetikzlibrary{positioning}
\usetikzlibrary{decorations}
\usetikzlibrary{patterns}

\usepackage{pgfplots}
\pgfplotsset{compat=1.18}

\title{El polinomio de Taylor}
\subtitle{Una introducción a la aproximación de funciones}
\author{Francisco González Rubio \and Pedro Pasalodos Guiral \and Mario Vago Marzal}
\date{Curso 2023--2024}
\institute{Universitat de València}

\begin{document}
  {
    % A workaround to hide an overfull \vbox warning produced by
    % the title frame of the Metropolis theme.
    \vfuzz=16pt
    \maketitle
  }

  \begin{frame}{Índice}
    \setbeamertemplate{section in toc}[sections numbered]
    \tableofcontents
  \end{frame}

  \section{El problema del péndulo}
    \begin{frame}{El problema del péndulo}
  \begin{center}
    \begin{tikzpicture}[
      scale=1.4,
      transform shape,
      every node/.style={font=\tiny},
    ]
      \begin{scope}[yshift=5.25cm]
        \draw[
          pattern=north east lines,
          pattern color=TolDarkBlue,
        ] (-2.5,-0.25) rectangle (2.5,0.25);
      \end{scope}
  
      \begin{scope}[yshift=5cm]
        \draw[dashed, ultra thin] 
          (0,-2.15) -- (2,-2.15);
  
        \draw[dotted, thick, draw=TolLightGreen] 
          (-120:2.15) arc (-120:-60:2.15);
  
        \draw (0,0) -- (-60:2)
          node[midway, right] {$R$};
        \draw[fill=TolLightGreen] 
          (-60:2.15) circle (0.15);
  
        \draw[dashed, thin] (0,0) -- (-90:2.15);
  
        \draw[->, draw=TolLightBrown] 
          (-90:0.5) arc (-90:-60:0.5)
          node[midway, below, ] {$\theta$};
  
        \begin{scope}[xshift=0.25cm]
          \draw[|-|]
            (-60:2.15) -- (-60:2.15 |- -2.15,-2.15)
            node[midway,right,inner sep=0pt,xshift=0.1cm]
              {$h = R(1 - \cos \theta)$};
        \end{scope}
      \end{scope}
    \end{tikzpicture}
  \end{center}

  \vfill

  \begin{center}
    \begin{tikzpicture}[
      scale=0.9,
      transform shape,
      every node/.style={
        font=\small,
      },
    ]
      \begin{scope}[
        every node/.style={
          minimum height=1.1cm,
          rectangle,
        },
      ]
        \node[fill=TolLightRed!20!white]
          (exacta) at (0,0) {
          $h = R(1 - \cos \theta)$
        };

        \node[fill=TolLightGreen!20!white] 
          (aprox) [right=4cm of exacta] {
          $h \approx R \dfrac{\theta^2}{2}$
        };
      \end{scope}

      \draw[decorate, decoration={snake, amplitude=0.4mm}, <->] (exacta) -- (aprox)
        node[midway, below, yshift=-0.3cm] {
          $\cos \theta \approx 1 - \dfrac{\theta^2}{2}$
        };
    \end{tikzpicture}
  \end{center}
\end{frame}


  \section{El polinomio de Taylor}

  \section{El error de aproximación}

  \section{Caracterización de los polinomios de Taylor}

  \section*{Conclusiones}
\end{document}

\begin{frame}{Primera caracterización}
  \begin{exampleblock}{Primer caracterización del polinomio de Taylor}
    Sea $f$ una función $n+1$ veces derivable en un entorno de $x_0$.
    El polinomio de Taylor de $f$ en $x_0$ de grado $n$ es el único
    polinomio $P_n(x)$ tal que
    \[
      \lim_{x \to x_0} \frac{f(x)-P_n(x)}{(x-x_0)^n} = 0.
    \]
  \end{exampleblock}

  Intuitivamente, lo que decimos es que a medida que nos acercamos a $x_0$,
  el polinomio $P_n(x)$ se parece cada vez más a $f(x)$.
\end{frame}

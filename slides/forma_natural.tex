\begin{frame}{Forma natural del polinomio de Taylor}
  Si tomamos la base natural de $\Pi_n$, obtenemos la forma natural del
  polinomio de Taylor:
    
  \begin{center}
    \begin{tikzpicture}
      \node[rectangle, fill=TolLightBlue!20!white] {
        $\displaystyle P_n(x) = 
        \sum_{i=0}^n \frac{f^{(i)}(x_0)}{n!} (x-x_0)^i$
      };
      \end{tikzpicture}
  \end{center}

  \begin{multicols}{2}
    Tomando $f(x) = \sen x$ y $x_0 = 0$, tenemos que
    \begin{itemize}
      \item $P_1(x) = x$,
      \item $P_3(x) = x-\frac{x^3}{6}$,
      \item y $P_5(x) = x - \frac{x^3}{6} + \frac{x^5}{120}$.
    \end{itemize}

    \columnbreak

    \begin{center}
      \begin{tikzpicture}[scale=0.6, transform shape]
        \begin{axis}[
          mlineplot,
          no markers,
          axis lines=center,
          legend pos=south east,
          trig format=rad,
          thick,
        ]
          \addplot+[
            domain=-pi:pi,
            samples=100,
          ] {sin(x)};
            \addlegendentry{$\sen(x)$}
  
          \addplot+[
              domain=-pi:pi,
              samples=100,
            ] {x};
              \addlegendentry{$P_1(x)$}
      
          \addplot+[
            domain=-pi:pi,
            samples=100,
          ] {x - x^3/6};
            \addlegendentry{$P_3(x)$}
      
          \addplot+[
            domain=-pi:pi,
            samples=100,
          ] {x - x^3/6 + x^5/120};
            \addlegendentry{$P_5(x)$}
        \end{axis}
      \end{tikzpicture}
    \end{center}
  \end{multicols}
\end{frame}

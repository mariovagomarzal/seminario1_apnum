\begin{frame}{Forma natural}
  \begin{block}{Forma natural}

    El polinomio de Taylor centrado en $x_0$ de grado $n$ se pude 
    escribir de la manera siguiente:
      
    \begin{center}
      \begin{tikzpicture}
        \node[rectangle, fill=TolLightBlue!20!white] {
          $\displaystyle P_n(x) = \sum_{i=0}^n \frac{f^{(i)}(x_0)}{n!} (x-x_0)^i$
          };
        \end{tikzpicture}
    \end{center}
  \end{block}

  \begin{multicols}{2}
    \begin{tikzpicture}[scale=0.6, transform shape]
      \begin{axis}[
        mlineplot,
        no markers,
        axis lines=center,
        legend pos=south east,
        trig format=rad,
        ]
        \addplot+[
          domain=-pi:pi,
          samples=100,
        ] {sin(x)};
          \addlegendentry{$\sen(x)$}
    
        \addplot+[
          domain=-pi:pi,
          samples=100,
        ] {x - x*x*x/6};
          \addlegendentry{$P_3(x)$}
    
        \addplot+[
            domain=-pi:pi,
            samples=100,
          ] {x - x*x*x/6 + x*x*x*x*x/120};
            \addlegendentry{$P_6(x)$}
         
      \end{axis}
    \end{tikzpicture}
  \columnbreak
  \begin{itemize}
    \item $P_1(x) = x$
    \item $P_3(x) = x-\frac{x^3}{6}$
    \item $P_5(x) = x - \frac{x^3}{6} + \frac{x^5}{120}$
  \end{itemize}
\end{multicols}
        

\end{frame}

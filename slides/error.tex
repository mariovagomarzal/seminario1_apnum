 \begin{frame}{Forma del error de Lagrange}
  \begin{center}
    \begin{tikzpicture}[scale=0.65, transform shape]
      \begin{axis}[
        mlineplot,
        no markers,
        axis lines=center,
        legend pos=south east,
        trig format=rad,
        thick,
      ]
        \addplot+[
          domain=-pi:pi,
          samples=100,
        ] {sin(x)};
          \addlegendentry{$\sen(x)$}
    
        \addplot+[
          domain=-2:2,
        ] {x};
          \addlegendentry{$x$}
        
        \def\errorx{pi/2}
        \node (val_real) at 
          (axis cs: \errorx, {sin(\errorx)}) {};
        \node (val_aprox) at 
          (axis cs: \errorx, \errorx) {};
        \draw[
          dashed,
          thin,
          draw=TolDarkRed,
          ->,
        ] (val_aprox.center) -- (val_real.center)
          node[midway, right] 
          {$\varepsilon_{\frac{\pi}{2}}$};
      \end{axis}
    \end{tikzpicture}
  \end{center}
  
  Si $f$ es $n + 1$ veces derivable, existe $\xi_x \in (x_0, x)$ tal que
  \begin{center}
    \begin{tikzpicture}
      \node[rectangle, fill=TolLightRed!20!white] {
        $\displaystyle f(x) - P_n (x) = \frac{f^{(n+1)} (\xi_x)}{(n + 1)!} (x - x_0)^{n+1}$
      };
    \end{tikzpicture}
  \end{center}
\end{frame}

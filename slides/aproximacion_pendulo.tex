\begin{frame}{La aproximación del péndulo}
  \begin{center}
    \begin{tikzpicture}[scale=0.55, transform shape]
      \node (coseno) {
        \begin{tikzpicture}
          \begin{axis}[
            mlineplot,
            no markers,
            axis lines=center,
            trig format=rad,
            thick,
          ]
            \addplot+[
              domain=-pi:pi,
              samples=100,
            ] {cos(x)};
              \addlegendentry{$\cos \theta$}
          \end{axis}
        \end{tikzpicture}
      };
      
      \node (taylor) [right=4cm of coseno] {
        \begin{tikzpicture}
          \begin{axis}[
            mlineplot,
            no markers,
            axis lines=center,
            trig format=rad,
            thick,
          ]
            \addplot+[
              domain=-pi:pi,
              samples=100,
              dashed,
              thin,
            ] {cos(x)};
              \addlegendentry{$\cos \theta$}
      
            \addplot+[
              domain=-2:2,
            ] {1 - x^2/2};
              \addlegendentry{$1 - \frac{\theta^2}{2}$}
          \end{axis}
        \end{tikzpicture}
      };
  
      \draw[
        decorate,
        decoration={snake, amplitude=0.4mm}, 
        <->,
      ]
        (coseno.east) -- (taylor.west)
        node [midway, below, yshift=-0.25cm] 
        {$\cos \theta \approx 1 - \dfrac{\theta^2}{2}$};
    \end{tikzpicture}
  \end{center}

    \[
      \cos \theta \approx
      1 - \frac{\theta^2}{2} + \frac{\theta^4}{4!} - \frac{\theta^6}{6!} + \cdots
    \]
\end{frame}

\begin{frame}{Segunda caracterización}
  \begin{exampleblock}{Segunda caracterización del polinomio de Taylor}
    Sea $f$ una función $n$ veces derivable en un entorno de $a$. El
    polinomio de Taylor de $f$ en $x_0$ de grado $n$ es el único polinomio
    $P_n(x)$ tal que
    \[
      \left\lvert f(x) - P_n(x) \right\rvert
      \le M \left\lvert x - x_0 \right\rvert^{n + 1},
    \]
    en un entorno de $x_0$, con $M > 0$.
  \end{exampleblock}

  Esta desigualdad nos recuerda la continuidad de Lipschitz. Análogamente a
  la otra caracterización, esta nos muestra que el polinomio de Taylor es
  el único polinomio que se aproxima tanto como queramos al comportamiento
  de $f$.
\end{frame}
